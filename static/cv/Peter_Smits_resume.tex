%May  resume.tex
%
% (c) 2002 Matthew Boedicker <mboedick@mboedick.org> (original author) http://mboedick.org
% (c) 2003 David J. Grant <dgrant@ieee.org> http://www.davidgrant.ca
% (c) 2007 Todd C. Miller <Todd.Miller@courtesan.com> http://www.courtesan.com/todd
% (c) 2009-2012 Derek R. Hildreth <derek@derekhildreth.com> http://www.derekhildreth.com 
% (c) 2019 Peter D. Smits <peterdavidsmits@gmail.com> https://psmits.github.io/
%This work is licensed under the Creative Commons Attribution-NonCommercial-ShareAlike License. To view a copy of this license, visit http://creativecommons.org/licenses/by-nc-sa/1.0/ or send a letter to Creative Commons, 559 Nathan Abbott Way, Stanford, California 94305, USA.

% GENERAL NOTE:  There may be some notes specific to myself.  If you're only interested in my LaTeX source or it doesn't make sense, please disregard it.

\documentclass[letterpaper,10pt]{article}

%-----------------------------------------------------------
\usepackage{latexsym}
%\usepackage[empty]{fullpage}
\usepackage[margin=0.5in,bottom=0mm,nofoot,nohead]{geometry}
\usepackage[usenames,dvipsnames]{color}
\usepackage{verbatim}
\usepackage[pdftex]{hyperref}
%\usepackage{paralist}
\usepackage{enumitem}
\hypersetup{
  colorlinks,%
  citecolor=black,%
  filecolor=black,%
  linkcolor=black,%
  urlcolor=black 
    %urlcolor=mygreylink     % can put red here to better visualize the links
}
\urlstyle{same}
\definecolor{mygrey}{gray}{.90}
\definecolor{mygreylink}{gray}{.40}
\textheight=10.0in
\raggedbottom
\raggedright
\setlength{\tabcolsep}{0in}

\pagenumbering{gobble}

% Adjust margins
%\addtolength{\oddsidemargin}{-0.375in}
%\addtolength{\evensidemargin}{0.375in}
%\addtolength{\textwidth}{0.5in}
%\addtolength{\topmargin}{-.375in}
%\addtolength{\textheight}{0.75in}

%-----------------------------------------------------------
%Custom commands
\newcommand{\resitem}[1]{\item #1 \vspace{-2pt}}
\newcommand{\resheading}[1]{
  {\large \colorbox{mygrey}{\begin{minipage}{\textwidth}{\textbf{#1 \vphantom{p\^{E}}}}\end{minipage}}}
}
\newcommand{\ressubheading}[4]{
  \begin{tabular*}{6.5in}{l@{\extracolsep{\fill}}r}
    \textbf{#1} & #2 \\
    \textit{#3} & \textit{#4} \\
\end{tabular*}\vspace{-6pt}}

% my implementation for eduction elements
\newcommand{\ressubheadingb}[3]{
  \begin{tabular*}{6.5in}{ll@{\extracolsep{\fill}}r}
    \textbf{#1} & \hspace{0.1mm} #2 & #3 \\
    %\textit{#3} & \textit{#4} \\
\end{tabular*}\vspace{-6pt}}

\newcommand{\ressubsubheading}[2]{
  \begin{tabular*}{6.5in}{l@{\extracolsep{\fill}}r}
    \textit{#1} & \textit{#2} \\
\end{tabular*}\vspace{-6pt}}
%-----------------------------------------------------------

%-----------------------------------------------------------
%General Resume Tips
%   No periods!  Technically, nothing in this document is a full sentence.
%   Use parallelism by ending key words with the same thing,  i.e. "Coordinated; Designed; Communicated".
%   More tips on bottom of this LaTeX document.
%-----------------------------------------------------------

\begin{document}

\newcommand{\mywebheader}{
  \begin{tabular*}{7in}{l@{\extracolsep{\fill}}r}
    \textbf{\href{https://psmits.github.io/}{\Large Peter David Smits}} & \href{https://psmits.github.io/}{psmits.github.io/} \\
    Seattle, Washington; 609-933-7042 & \href{https://github.com/psmits}{github.com/psmits} \\
    \href{mailto:peterdavidsmits@gmail.com}{peterdavidsmits@gmail.com} & \href{https://www.linkedin.com/in/pdsmits/}{linkedin.com/in/pdsmits/} \\
  \end{tabular*}
  \\
\vspace{0.05in}
}

% CHANGE HEADER SOURCE HERE
\mywebheader

%\resheading{Summary}
%  \begin{itemize}
%    \item Paleobiologist and data scientist with 8+ years research experience in evolution, ecology, and geology; works cited over 180 times
%    \item Bayesian data analysis and multilevel/hierarchical models across multiple applications (e.g. survival, longitudinal, and discrete time-series data; hidden Markov models)
%%    \item Six publications in peer-reviewed journals, cited 189 times
%%    \item International collaborator analyzing biomechanical, phylogenetic, and time-series count data and with applying machine learning techniques for prediction
%  \end{itemize}


%%%%%%%%%%%%%%%%%%%%%%
%\vspace{0.25in}
\resheading{Skills}
\begin{description}[itemsep=-1mm]
    %\begin{small}
    \item \textbf{Languages/Tools:}  R, Stan, Python, SQL, bash, \LaTeX, git, Docker
    \item \textbf{Packages:} \vspace{-5pt}
      \begin{itemize}[itemsep=-1mm]
        \item \textbf{R:} brms, rstan, rstanarm, lme4, ggplot2, tidyverse, tidymodels, rmarkdown/knitr, shiny
        \item \textbf{Python:} pandas, numpy, beautifulsoup, flask
      \end{itemize}
    \item \textbf{Statistics:} Bayesian statistics, multilevel/hierarchical modeling, survival analysis, longitudinal data analysis
    \item \textbf{Machine Learning:} regression, decision trees/CART, random forests, k-means, hierarchical clustering, regularization, feature engineering/selection, dimensionality reduction/PCA, neural networks
    %\end{small}
\end{description} % End Skills list
\vspace{0.05in}



%%%%%%%%%%%%%%%%%%%%%%
\resheading{Experience}
\begin{itemize}[itemsep=0mm]
  \item 
    \ressubheading{Data Scientist II}{Seattle, WA}{Amazon -- Advertising}{January 2020 -- present}
    { %normalsize%small%footnotesize
      \begin{itemize}
        \resitem{Developed Recommendation Impact, a data processing and modeling pipeline for measuring the effect of advertising campaign suggestions on daily advertising performance \textbf{R}, \textbf{Docker}, \textbf{ECR} and other AWS tools.}
        \resitem{Designed multilevel/mixed-effects models for describing effectiveness of automated advertising campaign setup suggestions on display advertisement performance, with Maximum Likelihood and Bayesian implementations, written in \textbf{R} using \textbf{lme4} and \textbf{Stan}.}
        %\resitem{Containerized and automated Recommendation Impact pipeline using \textbf{Docker}, \textbf{odbc}, \textbf{R}, and \textbf{bash}.}
        \resitem{Coordinated with international team to measure and report the quality and value of 24 automated display advertising performance recommendations.}
      \end{itemize}
    }
  \item 
    \ressubheading{Data Science Fellow}{Seattle, WA}{Insight Data Science}{Sept 2019 -- December 2019}
    { %normalsize%small%footnotesize
      \begin{itemize}
          \resitem{Designed Copyprism, a web application for generating draft product descriptions from an image.}
          \resitem{Implemented web scraper for collecting IKEA product descriptions in \textbf{python} using \textbf{beautifulsoup}.}
          \resitem{Fine-tuned GPT-2 natural language deep learning model on IKEA catalog in \textbf{python} whose generated text indistinguishable from human text 26\% of the time.}
          \resitem{Deployed Copyprism web application on AWS using \textbf{flask}.}
      \end{itemize}
    }
  \item 
    \ressubheading{Postdoctoral Scholar}{Berkeley, CA}{University of California -- Berkeley}{Sept 2017 -- July 2019}
    { %\normalsize%small%footnotesize
      \begin{itemize}
          \resitem{Designed multilevel survival model using \textbf{Stan} to predict species extinction which identified species at risk of extinction within 1-million years with an AUC of 0.78.}
          \resitem{Created multilevel Bayesian time series model in \textbf{Stan} for predicting when rare extinction events were likely to be clustered in time based on geological information across multiple databases.}
          \resitem{Wrote nine lesson \href{https://psmits.github.io/paleo_book/index.html}{short course} on analyzing paleontological and macroecological data using \textbf{R}, \textbf{tidyverse}, and \textbf{brms}  which was used as curriculum for graduate-level paleontology course.}
      \end{itemize}
    }
  \item 
    \ressubheading{Graduate Researcher}{Chicago, IL}{University of Chicago}{Sept 2012 -- June 2017}
    { %normalsize%small%footnotesize
      \begin{itemize}
          \resitem{Identified how differences in mammal species ecologies affected their survival rates over the last 65 million years using a multilevel Bayesian survival model implemented in \textbf{Stan} and \textbf{R} applied to a database of fossil occurrences accessed via web API.}
          \resitem{Created hidden Markov birth-death model in \textbf{Stan} for estimating fossil species observation, origination, and extinction rates over time, where each rate was modeled as its own multilevel regression which incorporated species ecology and environmental context.}
          %\resitem{Developed ensemble framework for identifying species based on 2D shape information using \textbf{R} and \textbf{caret} where average of multinomial logistic regression, neural network, and random forest models correctly differentiated seven closely related turtle species with an AUC over 0.98.}
          \resitem{Mentored and taught graduate and undergraduate students in R, statistics, Stan, and pedagogy.}
      \end{itemize}
    }
\end{itemize}  % End Experience list
\vspace{0.05in}


%%%%%%%%%%%%%%%%%%%%%%
\resheading{Projects}
\begin{itemize}[itemsep=-1mm]
  \item \textbf{mathhammr:} \textbf{R} package and \href{https://psmits.shinyapps.io/dice_roller/}{\textbf{shiny} web application} for generating dice rolls for Warhammer 40k 8th Edition and comparing those rolls to a simulated distribution of results.
\end{itemize}


%%%%%%%%%%%%%%%%%%%%%%
\resheading{Education}
\begin{itemize}
  \item
    \ressubheadingb{University of Chicago}{ Ph.D. Evolutionary Biology}{June 2017}
  \item
    \ressubheadingb{Monash University}{ M.Sc. Biological Sciences}{Aug 2012}
    \begin{itemize}
      \item {\small Vice-Chancellor's Commendation for Master's Thesis Excellence}
    \end{itemize} \vspace{-7.5pt}
  \item
    \ressubheadingb{University of Washington}{ B.S. Biology}{June 2010}
\end{itemize} % End Education list



%%%%%%%%%%%%%%%%%%%%%%

%\resheading{Publications}
%  \begin{description}
%    \item Stewart M Edie, {\bf Peter D Smits}, David Jablonski. Probabilistic models of species discovery and biodiversity comparisons. \emph{Proceedings of the National Academy of Sciences}, 114(14):3666-3671, 2016. {\bf IF 9.423}
%    \item {\bf Peter D Smits}. Expected time-invariant effects of biological traits on mammal species duration. \emph{Proceedings of the National Academy of Sciences}, 112(42):13015–13020, 2015. {\bf IF 9.423}
%    \item Liliana M Davalos, Paul M Velazco, Omar M Warsi, {\bf Peter D Smits}, Nancy B Simmons. Integrating Incomplete Fossils by Isolating Conflicting Signal in Saturated and Non-Independent Morphological Characters. \emph{Systematic Biology}, 63(4):582-600, 2014. {\bf IF 14.387}
%    \item Christopher W Walmsley, {\bf Peter D Smits}, Michelle R Quayle, Matthew R Mc-Curry, Heather S Richard, Christopher C Oldfield, Stephen Wroe, Phillip D Clausen, and Colin R McHenry. Why the fong face? The mechanics of mandibular symphysis proportions in crocodiles. \emph{PLoS ONE}, 8(1):e53873, 2013. {\bf IF 3.234}
%    \item {\bf Peter D Smits}, Alistair R Evans. Functional constraints on tooth morphology in carnivorous mammals. \emph{BMC Evolutionary Biology}, 12(1):146, 2012. {\bf IF 3.407}
%    \item Gregory P Wilson, Alistair R Evans, Ian J Corfe, {\bf Peter D Smits}, Mikael Fortelius, and Jukka Jernvall. Adaptive radiation of multituberculate mammals before the extinction of dinosaurs. \emph{Nature}, 483:457-460, 2012. {\bf IF 39.138}
%  \end{description}

%%%%%%%%%%%%%%%%%%%%%%

%\resheading{Selected Conference Presentations (last 3 years)}
%  \begin{description}
%    \item {\bf Speaker} Geological Society of America Annual Meeting, "Taxon occurrence as a function of both biological traits and environmental context: the changing North American species pool," Denver, CO, Sept 2016
%    \item {\bf Speaker} Geological Society of America Annual Meeting, "Taxon occurrence as a function of both biological traits and environmental context: the changing North American species pool," Denver, CO, Sept 2016
%    \item {\bf Speaker} Evolution Meetings, "How macroecology affects macroevolution: the interplay between extinction intensity and trait-dependent extinction in brachiopods," Austin, TX, June 2016
%    \item {\bf Speaker} Geological Society of America Annual Meeting, "How do biological traits affect brachiopod taxonomic survival? A hierarchical Bayesian approach," Baltimore, MD, Nov 2015
%    \item {\bf Speaker} Evolution Meetings, "Death and taxa: time-invariant differences in mammal species duration," Guaruja, Brazil, June 2015
%    \item {\bf Speaker} Geological Society of America Annual Meeting, "Gambling with Australian brachiopods," Vancouver, BC, Oct 2016
%    \item {\bf Speaker} Evolution Meetings, "Cenozoic mammals and the biology of extinction," Raleigh, NC, June 2014
%  \end{description}


\end{document}
