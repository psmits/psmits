% resume.tex
%
% (c) 2002 Matthew Boedicker <mboedick@mboedick.org> (original author) http://mboedick.org
% (c) 2003 David J. Grant <dgrant@ieee.org> http://www.davidgrant.ca
% (c) 2007 Todd C. Miller <Todd.Miller@courtesan.com> http://www.courtesan.com/todd
% (c) 2009-2012 Derek R. Hildreth <derek@derekhildreth.com> http://www.derekhildreth.com 
% (c) 2017 Peter D. Smits <peterdavidsmits@gmail.com> https://psmits.github.io/
%This work is licensed under the Creative Commons Attribution-NonCommercial-ShareAlike License. To view a copy of this license, visit http://creativecommons.org/licenses/by-nc-sa/1.0/ or send a letter to Creative Commons, 559 Nathan Abbott Way, Stanford, California 94305, USA.

% GENERAL NOTE:  There may be some notes specific to myself.  If you're only interested in my LaTeX source or it doesn't make sense, please disregard it.

\documentclass[letterpaper,11pt]{article}

%-----------------------------------------------------------
\usepackage{latexsym}
\usepackage[empty]{fullpage}
\usepackage[usenames,dvipsnames]{color}
\usepackage{verbatim}
\usepackage[pdftex]{hyperref}
\hypersetup{
  colorlinks,%
  citecolor=black,%
  filecolor=black,%
  linkcolor=black,%
  urlcolor=black 
    %urlcolor=mygreylink     % can put red here to better visualize the links
}
\urlstyle{same}
\definecolor{mygrey}{gray}{.85}
\definecolor{mygreylink}{gray}{.40}
\textheight=9.0in
\raggedbottom
\raggedright
\setlength{\tabcolsep}{0in}

% Adjust margins
\addtolength{\oddsidemargin}{-0.375in}
\addtolength{\evensidemargin}{0.375in}
\addtolength{\textwidth}{0.5in}
\addtolength{\topmargin}{-.375in}
\addtolength{\textheight}{0.75in}

%-----------------------------------------------------------
%Custom commands
\newcommand{\resitem}[1]{\item #1 \vspace{-2pt}}
\newcommand{\resheading}[1]{{\large \colorbox{mygrey}{\begin{minipage}{\textwidth}{\textbf{#1 \vphantom{p\^{E}}}}\end{minipage}}}}
\newcommand{\ressubheading}[4]{
  \begin{tabular*}{6.5in}{l@{\extracolsep{\fill}}r}
    \textbf{#1} & #2 \\
    \textit{#3} & \textit{#4} \\
\end{tabular*}\vspace{-6pt}}

\newcommand{\ressubsubheading}[2]{
  \begin{tabular*}{6.5in}{l@{\extracolsep{\fill}}r}
    \textit{#1} & \textit{#2} \\
\end{tabular*}\vspace{-6pt}}
%-----------------------------------------------------------

%-----------------------------------------------------------
%General Resume Tips
%   No periods!  Technically, nothing in this document is a full sentence.
%   Use parallelism by ending key words with the same thing,  i.e. "Coordinated; Designed; Communicated".
%   More tips on bottom of this LaTeX document.
%-----------------------------------------------------------

\begin{document}

\newcommand{\mywebheader}{
  \begin{tabular*}{7in}{l@{\extracolsep{\fill}}r}
    \textbf{\href{https://psmits.github.io/}{\LARGE Peter David Smits}} & \href{mailto:peterdavidsmits@gmail.com}{peterdavidsmits@gmail.com}\\
    {\footnotesize \texttt{ph: 609-933-7042; add: 1661 Hopkins St, Berkeley, CA 94707}} & \href{https://psmits.github.io/}{https://psmits.github.io/} \\
  \end{tabular*}
  \\
\vspace{0.1in}}

% CHANGE HEADER SOURCE HERE
\mywebheader

%\resheading{Summary}
%  \begin{itemize}
%    \item Paleobiologist and data scientist with 8+ years research experience in evolution, ecology, and geology; works cited over 180 times
%    \item Bayesian data analysis and multilevel/hierarchical models across multiple applications (e.g. survival, longitudinal, and discrete time-series data; hidden Markov models)
%%    \item Six publications in peer-reviewed journals, cited 189 times
%%    \item International collaborator analyzing biomechanical, phylogenetic, and time-series count data and with applying machine learning techniques for prediction
%  \end{itemize}

%%%%%%%%%%%%%%%%%%%%%%
\resheading{Experience}
\begin{itemize}
  \item 
    \ressubheading{University of California -- Berkeley}{Berkeley, CA}{Postdoctoral Scholar}{Sept 2017 -- present}
    { \footnotesize
      \begin{itemize}
          \resitem{Executed multiple projects about predicting when rare events (e.g. extinctions) are likely to be clustered }
          \resitem{Developed a discrete-time survival model for predicting extinction from highly structured data }
          \resitem{Used results from this survival model to estimate the accuracy of predicting future extinction events }
          \resitem{Analyzed how sedimentological composition predicts the occurrence of rare events (e.g. fossils) using a time-series mixture model }
          \resitem{Taught graduate level course on applied statistics, Bayesian modeling, and network analysis using R and Stan }
          \resitem{Elected to University of California Postdoctoral Union Joint Council as Recording Secretary for UC Berkeley }
      \end{itemize}
    }
  \item 
    \ressubheading{University of Chicago}{Chicago, IL}{Doctoral Researcher}{Sept 2012 -- June 2017}
    { \footnotesize
      \begin{itemize}
          \resitem{Completed dissertation on Bayesian modeling of multi-level variation in species duration with incomplete observation }
          \resitem{Analyzed cohort-structured survival data using a multi-level model accounting for multiple forms of nonindependence in the observations }
          \resitem{Developed a hidden Markov model of species duration and their occurrences over time with incomplete observation probability }
          \resitem{Mentored and taught graduate and undergraduate students in statistics, Stan, R, and pedagogy }
      \end{itemize}
    }
		\item 
			\ressubheading{Monash University}{Melbourne, AUS}{Postgraduate Researcher}{Sept 2010 -- Aug 2012}
				{ \footnotesize
				\begin{itemize}
            \resitem{Developed thesis on the relationship between tooth shape and jaw movement in carnivorous mammals }
            \resitem{Used three-dimenstional scans of mammal skulls to reconstruct jaw movement }
            \resitem{Used R to analyze relationship among biomechanical measurements }
            \resitem{Demonstrated for undergraduate course on introduction to data analysis and R }
				\end{itemize}
				}
%		\item 
%			\ressubheading{Macquarie University}{Sydney, AUS}{Paleobiology Database Intensive Workshop in Quantitative Paleobiology}{July 2011 -- Aug 2011}
%				{ \footnotesize
%				\begin{itemize}
%            \resitem{5 week intensive workshop on statistical and analytical methods and their applications in paleobiology}
%            \resitem{Trained in using R for birth-death modelling, basic Gaussian processes, exploratory multivariate data analysis, phylogentic comparative methods, and geometric morphometrics}
%				\end{itemize}
%				}
%    \item 
%      \ressubheading{Burke Museum of Natural History}{Seattle, WA}{Collections Assistant in Paleontology}{June 2010 -- Aug 2010}
%        { \footnotesize
%        \begin{itemize}
%            \resitem{Catalogued specimens in the collections which appeared in publication}
%            \resitem{Managed specimen database growth and annotation}
%        \end{itemize}
%        }
%    \item 
%      \ressubheading{American Museum of Natural History}{New York, NY}{NSF Research Experience for Undergraduates Internship}{June 2009 -- Aug 2009}
%        { \footnotesize
%        \begin{itemize}
%            %\resitem{Collected data describing bat tooth shapes }
%            \resitem{Analyzed similarities in bat tooth shape in a phylogenetic context, resulted in one publication }
%        \end{itemize}
%        }
%    \item 
%      \ressubheading{University of Washington}{Seattle, WA}{Undergraduate Research Assistant}{Jan 2008 -- June 2010}
%        { \footnotesize
%        \begin{itemize}
%            \resitem{Analyzed relationship between mammal tooth size and body size using linear regression in order to predict body sizes of extinct mammals, resulted in one publication }
%            %\resitem{Assisted with field collection of fossils from the Hell Creek Formation (Latest Cretaceous) in Montana }
%        \end{itemize}
%        }
%    \item 
%      \ressubheading{Burke Museum of Natural History}{Seattle, WA}{Field and Lab Assistant in Mammalogy}{Oct 2006 -- Aug 2008}
%        { \footnotesize
%        \begin{itemize}
%            \resitem{Assisted with museum specimen preparation, taxidermy, and curation}
%            \resitem{Assisted with wild specimen collection and preparation}
%        \end{itemize}
%        }
\end{itemize}  % End Experience list

%%%%%%%%%%%%%%%%%%%%%%
\resheading{Education}
\begin{itemize}
  \item
    \ressubheading{University of Chicago}{Chicago, IL}{Ph.D. in Evolutionary Biology}{Sept 2012 -- June 2017}
  \item
    \ressubheading{Monash University}{Melbourne, AUS}{M.Sc. in Biological Sciences}{Sept 2010 -- Aug 2012}
    \begin{itemize}
      \item Vice-Chancellor's Commendation for Master's Thesis Excellence
    \end{itemize}
  \item
    \ressubheading{University of Washington}{Seattle, WA}{B.S. in Biology -- Ecology and Evolution}{Sept 2006 -- June 2010}
\end{itemize} % End Education list


%%%%%%%%%%%%%%%%%%%%%%

%\resheading{Projects}
%
%\begin{description}
  %  \item[:] { }
%\end{description}

%%%%%%%%%%%%%%%%%%%%%%
\vspace{0.25in}
\resheading{Technical Skills}
\begin{description}
  \item[Statistical/Analytical:] { \footnotesize Bayesian data analysis, multilevel/hierarchical/mixed-effects models, generalized linear models, time-series analysis, survival analysis, longitudinal and cross-sectional data analysis, discrete-time hidden Markov models, network analysis/graph theory, exploratory data analysis, clustering and classification, machine learning (e.g. random forests), measurement error/missing data, variable selection, etc. }
  \item[Technologies:] { \footnotesize R (caret, devtools, ggplot2, knitr, igraph, parallel, tidyverse, shiny), Stan, JAGS, \LaTeX, git/github, bash/Linux command line }
  \item[Other:] { \footnotesize near-fluency French, dual US--Australian citizen, radio experience }
\end{description} % End Skills list
\vspace{0.25in}

%%%%%%%%%%%%%%%%%%%%%%

%\resheading{Publications}
%  \begin{description}
%    \item Stewart M Edie, {\bf Peter D Smits}, David Jablonski. Probabilistic models of species discovery and biodiversity comparisons. \emph{Proceedings of the National Academy of Sciences}, 114(14):3666-3671, 2016. {\bf IF 9.423}
%    \item {\bf Peter D Smits}. Expected time-invariant effects of biological traits on mammal species duration. \emph{Proceedings of the National Academy of Sciences}, 112(42):13015–13020, 2015. {\bf IF 9.423}
%    \item Liliana M Davalos, Paul M Velazco, Omar M Warsi, {\bf Peter D Smits}, Nancy B Simmons. Integrating Incomplete Fossils by Isolating Conflicting Signal in Saturated and Non-Independent Morphological Characters. \emph{Systematic Biology}, 63(4):582-600, 2014. {\bf IF 14.387}
%    \item Christopher W Walmsley, {\bf Peter D Smits}, Michelle R Quayle, Matthew R Mc-Curry, Heather S Richard, Christopher C Oldfield, Stephen Wroe, Phillip D Clausen, and Colin R McHenry. Why the fong face? The mechanics of mandibular symphysis proportions in crocodiles. \emph{PLoS ONE}, 8(1):e53873, 2013. {\bf IF 3.234}
%    \item {\bf Peter D Smits}, Alistair R Evans. Functional constraints on tooth morphology in carnivorous mammals. \emph{BMC Evolutionary Biology}, 12(1):146, 2012. {\bf IF 3.407}
%    \item Gregory P Wilson, Alistair R Evans, Ian J Corfe, {\bf Peter D Smits}, Mikael Fortelius, and Jukka Jernvall. Adaptive radiation of multituberculate mammals before the extinction of dinosaurs. \emph{Nature}, 483:457-460, 2012. {\bf IF 39.138}
%  \end{description}

%%%%%%%%%%%%%%%%%%%%%%

%\resheading{Selected Conference Presentations (last 3 years)}
%  \begin{description}
%    \item {\bf Speaker} Geological Society of America Annual Meeting, "Taxon occurrence as a function of both biological traits and environmental context: the changing North American species pool," Denver, CO, Sept 2016
%    \item {\bf Speaker} Geological Society of America Annual Meeting, "Taxon occurrence as a function of both biological traits and environmental context: the changing North American species pool," Denver, CO, Sept 2016
%    \item {\bf Speaker} Evolution Meetings, "How macroecology affects macroevolution: the interplay between extinction intensity and trait-dependent extinction in brachiopods," Austin, TX, June 2016
%    \item {\bf Speaker} Geological Society of America Annual Meeting, "How do biological traits affect brachiopod taxonomic survival? A hierarchical Bayesian approach," Baltimore, MD, Nov 2015
%    \item {\bf Speaker} Evolution Meetings, "Death and taxa: time-invariant differences in mammal species duration," Guaruja, Brazil, June 2015
%    \item {\bf Speaker} Geological Society of America Annual Meeting, "Gambling with Australian brachiopods," Vancouver, BC, Oct 2016
%    \item {\bf Speaker} Evolution Meetings, "Cenozoic mammals and the biology of extinction," Raleigh, NC, June 2014
%  \end{description}


\end{document}
